\documentclass[fleqn,10pt]{wlscirep}
\newcommand{\autocite}[1]{\cite{#1}}
\newcommand{\textcite}[1]{\cite{#1}}

\title{Trabajo 1}

%%------------AUTHORS--------------
\author[50201633Q / jgarcia1285 ]{Javier Garcia Parra}


%%------------/AUTHORS--------------


%\keywords{Keyword1, Keyword2, Keyword3}

\begin{abstract}
%% Text of abstract
Este es el primer trabajo de la asignatura `Herramientas informáticas
para las matemáticas' del grado en matemáticas de la UNED, curso
2019/20.
\end{abstract}


\begin{document}

\flushbottom
\maketitle
% * <john.hammersley@gmail.com> 2015-02-09T12:07:31.197Z:
%
%  Click the title above to edit the author information and abstract
%
\thispagestyle{empty}

%\noindent Please note: Abbreviations should be introduced at the first mention in the main text – no abbreviations lists. Suggested structure of main text (not enforced) is provided below.

%% main text
\section{Introducción}\label{introducciuxf3n}

El enunciado del trabajo es el siguiente:\\
Se desea estudiar cómo será el aterrizaje de los aviones en un
aeropuerto de nuevo diseño, situado cerca de una colina. La trayectoria
del aterrizaje, que tendrá forma sigmoidal como muestra la Figura 1.1,
se iniciará a una altura A {[}m{]}, el doble de la altura de la colina.
Y se puede describir mediante la siguiente ecuación.

\[y={{A\,e^ {- {{c\,x}\over{v_{0}}} }}\over{e^ {- {{c\,x}\over{v_{0}}} }+1}}\]

\section{Métodos}\label{muxe9todos}

Para la resolución se han empleado las herramientas maxima 5.41 y scilab
6.0.1.

\section{Resultados}\label{resultados}

\subsection{Apartados a resolver con
Maxima}\label{apartados-a-resolver-con-maxima}

\emph{1.1) Determinar el punto de la trayectoria donde el avión alcanza
la mínima inclinación.\\
1.2) Analizar el efecto de la velocidad inicial, v , en la trayectoria.
Representándola gráficamente para 4 valores de v pertenece a
{[}55,220{]} . Con A 500, c 0.2777 y x pertenece a {[}5000,5000{]}.\\
1.3) Por normativa de seguridad, la inclinación del avión durante la
trayectoria no puede ser inferior a -14 grados. Calcular, si A 500 y c
0.2777, la velocidad inicial v mínima con la que debe iniciar la
maniobra de aterrizaje el avión para cumplir esa condición.\\
1.4) Calcular la posición vertical del avión si éste avanzara hacia una
pista infinitamente larga.}

Para la ejecución del código, arrancar el intérprete de comandos de
maxima y ejecutar:

\begin{verbatim}
load("maxima.mac");
\end{verbatim}

Tras un tiempo, que puede alcanzar varios minutos, se alcanzan las
soluciones. El propio script muestra el apartado de la respuesta en el
que se encuentra.\\
Cabe mencionar que para el cálculo del punto de la trayectoria, se ha
tomado como origen de la maniobra de descenso el punto en el que el
avión ha descendido 10 m (y=490) (siguiendo el mismo criterio del
apartado 1.5, en el que se considera el contacto con la pista en y =
10.\\
Una vez finalizada la ejecución, se debería ver el gráfico pedido en el
apartado 1.2. Opcionalmente se puede visualizar la primera y segunda
derivada (empleando la función \texttt{plot\_inclinacion()}).\\
Para la aproximación de valores de la función se ha probado con los
algoritmos de optimización vistos en el curso, pero finalmente se ha
optado por emplear la función \texttt{solve()} nativa.

\subsection{Apartados a resolver con
Scilab}\label{apartados-a-resolver-con-scilab}

\emph{1.5) Calcular, con A 500, c 0.2777 y v 150 , el valor aproximado
de x donde el avión contacta con la pista, si se sabe que esto ocurre
cuando y 10. Y comprobarlo gráficamente.} Desde Scilab ejecutar el
script \texttt{scilab\_5.sce}. La solución debería aparecer en la
consola y el gráfico con la comprobación gráfica debería abrirse en una
nueva ventana.\\
\emph{1.6) Una vez que el avión ha contactado con la pista, se inicia la
maniobra de frenado. Su posición en la pista, x, depende del tiempo, t,
y se puede representar mediante la siguiente expresión. Donde a {[}m/s 2
{]} es la aceleración de frenado, v {[}m/s{]} es la velocidad del avión
en el instante de contactar con la pista y x es la posición de contacto
calculada en el apartado anterior. Representar gráficamente la posición
del avión en la pista en función del tiempo, si a 12, para 3 valores de
v (50, 75, 100) es decir para tres valores de velocidad inferiores a la
velocidad que tenía el avión al iniciar la maniobra de aterrizaje.\\
1.7) Calcular, en las tres situaciones del apartado 1.6, el tiempo que
necesita el avión para frenar completamente (velocidad nula) y la
distancia recorrida en la pista.}

Desde Scilab ejecutar el script \texttt{scilab\_6\_7.sce}. Previamente
se ha debido ejecutar el script del apartado 1.5, parte de cuyos
resultados son necesarios aquí.\\
La solución debería aparecer en la consola y el gráfico con la
comprobación gráfica debería abrirse en una nueva ventana.





\end{document}
